\section{Постановка задачи}

\subsection{Требования к хранимым данным}

Каждая запись в телефонном справочнике представляется объектом класса \texttt{Contact}, который содержит набор полей для хранения информации о пользователе. В состав данных входят фамилия, имя, отчество, адрес проживания, дата рождения, адрес электронной почты и список телефонных номеров. Для строковых данных используется тип \texttt{QString}, для даты рождения --- тип \texttt{QDate}, для набора телефонов --- контейнер \texttt{QStringList}. 

Фамилия, имя и отчество должны начинаться с заглавной буквы, а последующие буквы должны быть маленькими и содержать только допустимые символы. Адрес проживания хранится в виде строки без ограничений на формат, но с удалением лишних пробелов и при наличии символа <<|>> также происходит его удаление, дабы не перечить загрузке контакта из файла, где используется этот разделительный символ. Дата рождения должна быть корректной и меньше текущей даты хотя бы на один день. Электронная почта должна включать имя пользователя, символ «@», доменное имя и доменную зону. Телефонные номера допускают национальный и международный формат, при сохранении они нормализуются до последовательности цифр и символа «+». Минимальная длина нормализованного номера составляет десять символов. 

Все данные сохраняются в текстовом файле, где каждая строка соответствует одному контакту. Поля разделяются символом «|». Для корректной загрузки и сохранения запрещено использование разделителя внутри значений полей.


\subsection{Функциональные требования}

Приложение должно реализовывать следующие функциональные возможности.

\vspace{1 em}

\textbf{Добавление нового контакта.}\\ \\ Пользователь должен иметь возможность добавить новый контакт в справочник. Форма ввода должна содержать поля для всех обязательных данных контакта. После успешного добавления контакт должен появиться в табличном представлении.

\vspace{1 em}

\textbf{Редактирование существующего контакта.}\\ \\ Пользователь должен иметь возможность изменить данные выбранного контакта. После внесения изменений и их валидации обновленные данные должны отразиться в таблице.

\vspace{1 em}

\textbf{Удаление контакта.}\\ \\ Пользователь должен иметь возможность удалить выбранный контакт из справочника. Кнопка удаления должна быть активна только при выбранной строке в таблице. После удаления контакт должен исчезнуть из табличного представления.

\vspace{1 em}

\textbf{Поиск и фильтрация контактов.}\\ \\ Пользователь должен иметь возможность найти контакт по введённому запросу и выбранному полю. Поиск должен выполняться по каждому полю таблицы. Фильтрация должна происходить в реальном времени при вводе текста. При очистке поля поиска должны отображаться все контакты.

\vspace{1 em}

\textbf{Сортировка данных.}\\ \\ Пользователь должен иметь возможность упорядочить контакты по любому полю. Сортировка должна выполняться через клик по заголовку столбца в таблице. При этом после сортировки должен корректно запоминаться индекс уже созданного контакта, чтобы впоследствии при редактировании и удалении выбирался нужный.

\vspace{1 em}

\textbf{Сохранение и загрузка данных.}\\ \\ Данные справочника должны сохраняться на локальное устройство и загружаться. После каждой операции добавления, редактирования или удаления данных должно реализовываться корректное сохранение в файл, путь которого указан в программе.

\subsection{Требования к валидации данных}

Все вводимые пользователем данные должны проходить проверку на соответствие заданным правилам с использованием регулярных выражений. \\

\textbf{Валидация имени, фамилии и отчества.}\\ \\ Длина строки должна быть больше нуля после удаления пробелов по краям. Первый символ должен быть заглавной буквой кириллицы или латиницы, последующие обязательно должны быть записаны с маленькой буквы. Допустимые символы включают буквы различных алфавитов, пробелы, дефисы и цифры. Строка не может начинаться или заканчиваться дефисом. Пробелы в начале и конце строки должны автоматически удаляться. \\

\textbf{Валидация телефонного номера.}\\ \\ Допускается международный формат с символом плюс или национальный формат. Допустимые символы включают цифры, символ плюс, пробелы. Примеры допустимых форматов: \texttt{+79287766000}, \texttt{89287766000}. При сохранении телефон нормализуется до последовательности цифр и символа плюс. Минимальная длина нормализованного номера составляет 10 символов. \\

\textbf{Валидация даты рождения.}\\ \\ Дата должна быть валидной, что проверяется через QDate::isValid. Дата должна быть строго меньше текущей даты. Проверка выполняется сравнением с QDate::currentDate. Для ввода используется виджет QDateEdit с форматом отображения DD.MM.YYYY. Виджет имеет всплывающий календарь, активируемый через setCalendarPopup. \\

\textbf{Валидация Email.}\\ \\ Формат должен соответствовать схеме username@domain.zone. Имя пользователя должно состоять из одной или более латинских букв или цифр. Обязательно наличие символа @. Доменное имя должно состоять из одной или более латинских букв или цифр. Обязательно наличие точки после домена. Зона домена должна содержать одну или более латинских букв или цифр. Пробелы в строке должны автоматически удаляться.

\subsection{Требования к пользовательскому интерфейсу}

Таблица должна поддерживать автоматическую сортировку по столбцам. Режим выбора должен позволять выделение только целых строк через SelectRows.

Форма ввода и редактирования реализуется через отдельное диалоговое окно ContactDialog, которое содержит все необходимые поля. Кнопки управления располагаются в нижней части главного окна. Состояние кнопок зависит от текущего контекста, например, кнопки редактирования, удаления активны только при выборе строки в таблице.