\section{Постановка задачи}
\label{sec:task}

Целью работы является разработка программного приложения <<Телефонный справочник>>, обеспечивающего хранение, обработку и управление контактной информацией пользователей с использованием локальной базы данных в виде текстового файла.
\\ \\
Для достижения поставленной цели необходимо решить следующие задачи:

\begin{enumerate}
    \item \textbf{Определить структуру данных для хранения информации о контакте.}\\  
    Требуется разработать класс, включающий следующие поля: фамилия, имя, отчество, адрес проживания, дата рождения, адрес электронной почты и список телефонных номеров.  
    Структура должна обеспечивать строгую валидацию каждого поля, преобразование объекта в строку для сохранения в файл и восстановление объекта из строки при загрузке.

    \item \textbf{Разработать формат хранения данных.}\\
    Необходимо определить единый текстовый файл, где каждая строка соответствует одному контакту, выбрать фиксированный разделитель полей, задать правила нормализации данных (удаление лишних пробелов), а также запретить использование разделителя внутри значений.

    \item \textbf{Реализовать механизм загрузки данных из файла.}\\
    Программа должна открывать файл, считывать строки до конца файла, корректно обрабатывать пустые строки и ошибки формата, создавать объекты на основе строк и формировать внутренний список контактов.

    \item \textbf{Реализовать механизм сохранения данных в файл.}\\
    Требуется сохранять контакты после операции изменения, записывать данные в текстовом виде, гарантировать корректность формата.

    \item \textbf{Реализовать функциональность добавления контактов.}\\
    Необходимо создать диалоговое окно ввода данных, выполнить валидацию всех полей, добавить новый контакт в список, обновить таблицу.

    \item \textbf{Реализовать функциональность редактирования контактов.}\\
    Программа должна корректно определять реальный индекс контакта даже после сортировки, открывать диалог с предзаполненными данными, выполнять повторную валидацию.

    \item \textbf{Реализовать функциональность удаления контактов.}\\
    Требуется определить выбранную строку, запросить подтверждение удаления, удалить соответствующий объект из списка, обновить таблицу.

    \item \textbf{Разработать пользовательский интерфейс для работы со справочником.}\\
    Интерфейс должен обеспечивать отображение контактов в табличном виде, выбор строк, сортировку по каждому столбцу, поиск по выбранному полю и динамическое скрытие / отображение строк при фильтрации.

    \begin{itemize}
    \item \textbf{Реализовать сортировку данных.}\\
    Пользователь должен иметь возможность упорядочить контакты по любому полю. Сортировка должна выполняться через клик по заголовку столбца в таблице. При этом после сортировки должен корректно запоминаться индекс уже созданного контакта, чтобы впоследствии при редактировании и удалении выбирался нужный.
    
    \item \textbf{Обеспечить корректную работу поиска и фильтрации.}\\
    Программа должна выполнять поиск по выбранному столбцу, учитывать особенности поля телефонов (несколько значений), обновлять отображение таблицы в реальном времени.
    \end{itemize}

    \item \textbf{Реализовать валидацию всех полей контакта.}\\
    Необходимо формализовать требования к каждому типу данных и обеспечить проверку корректности пользовательского ввода до сохранения информации.\\ \\
    Требования включают:

    \begin{itemize}
    \item \textbf{Валидация имени, фамилии и отчества.}\\
    Длина строки после удаления пробелов по краям должна быть больше нуля. Первый символ обязан быть заглавной буквой кириллицы или латиницы, последующие - строчными. Допустимые символы: буквы, пробелы, дефисы, цифры. Строка не может начинаться или заканчиваться дефисом. Пробелы в начале и конце должны удаляться автоматически.

    \item \textbf{Валидация телефонного номера.}\\
    Допускается международный формат с символом <<+>> или национальный формат. Разрешённые символы: цифры, символ <<+>>, пробелы. Примеры корректных форматов: \texttt{+79287766000}, \texttt{89287766000}. При сохранении номер должен нормализоваться до последовательности цифр и символа <<+>>. Должна быть реализована проверка минимальной длины нормализованного номера.
    
    \item \textbf{Валидация даты рождения.}\\
    Дата должна быть валидной и строго меньше текущей даты. Формат ввода должен обеспечивать корректность выбора даты пользователем.
    
    \item \textbf{Валидация адреса электронной почты.}\\
    Формат должен соответствовать схеме \texttt{username@domain.zone}. Имя пользователя — одна или более латинских букв или цифр. Обязательно наличие символа <<@>>. Доменное имя — одна или более латинских букв или цифр. После домена обязательно должна следовать точка. Доменная зона — одна или более латинских букв или цифр. Пробелы в строке должны автоматически удаляться.
    \end{itemize}
\end{enumerate}
