\section*{Заключение}

В ходе выполнения курсовой работы была разработана программная система
«Телефонный справочник», реализованная с использованием фреймворка Qt.
Работа охватывала полный цикл создания приложения: от
проектирования структуры данных до разработки пользовательского
интерфейса, механизмов валидации и тестирования. Ниже приведены
результаты по каждому пункту постановки задачи.

\textbf{1. Разработана структура данных для хранения информации о контакте.}
Создан класс \texttt{Contact}, включающий поля фамилии, имени, отчества,
адреса, даты рождения, электронной почты и списка телефонных номеров.
Реализованы методы \texttt{toString()} и \texttt{fromString()} для
конвертации объекта в строку и восстановления данных при загрузке.

\textbf{2. Определён формат хранения данных.}
Выбран текстовый файл с фиксированным разделителем полей. Реализована
нормализация строк, удаление лишних пробелов и запрет использования
разделителя внутри значений. Формат полностью поддерживается методами
класса \texttt{Contact}.

\textbf{3. Реализован механизм загрузки данных из файла.}
Метод \texttt{loadFromFile()} класса \texttt{PhoneBook} открывает файл,
читает строки до конца, корректно обрабатывает пустые и некорректные
строки, создаёт объекты \texttt{Contact} и формирует внутренний список
контактов.

\textbf{4. Реализован механизм сохранения данных в файл.}
Метод \texttt{saveToFile()} записывает текущий список контактов в файл в
строгом текстовом формате. Используется поток \texttt{QTextStream} с
кодировкой UTF--8, что обеспечивает корректность хранения данных.

\textbf{5. Реализована функциональность добавления контактов.}
Метод \texttt{addContact()} открывает диалог \texttt{ContactDialog},
выполняет валидацию данных и добавляет новый объект \texttt{Contact} в
список. После добавления вызывается \texttt{updateTable()} для обновления
интерфейса.

\textbf{6. Реализована функциональность редактирования контактов.}
Метод \texttt{editContact()} корректно определяет реальный индекс
контакта через \texttt{UserRole}, открывает диалог с предзаполненными
данными и выполняет повторную валидацию. После подтверждения изменения
сохраняются и отображаются в таблице.

\textbf{7. Реализована функциональность удаления контактов.}
Метод \texttt{deleteContact()} удаляет выбранный контакт из списка,
предварительно запрашивая подтверждение через \texttt{QMessageBox}.
Удаление корректно отражается в таблице благодаря вызову
\texttt{updateTable()}.

\textbf{8. Разработан пользовательский интерфейс.}
Создано главное окно \texttt{PhoneBook} с таблицей \texttt{QTableWidget},
панелью поиска, кнопками управления и поддержкой сортировки по
столбцам. Реализован механизм поиска по выбранному полю методом
\texttt{searchByColumn()}, включая обработку нескольких телефонных
номеров в одной ячейке.

\textbf{9. Реализована валидация всех полей контакта.}
В классе \texttt{ContactDialog} разработаны методы \texttt{validateName()},
\texttt{validatePhone()} и \texttt{validateEmail()}, использующие
регулярные выражения и дополнительные проверки (корректность регистра,
запрет дефиса в начале/конце, проверка даты рождения через \texttt{QDate}).
Валидация выполняется перед созданием объекта \texttt{Contact}.

Таким образом, все задачи, поставленные в начале работы, были полностью
выполнены. Разработанное приложение демонстрирует принципы
объектно-ориентированного программирования, работу с пользовательским
вводом, файловыми операциями, регулярными выражениями и элементами
графического интерфейса Qt.
