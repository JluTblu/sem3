\section{Математическое описание}

\subsection{Малая и большая конечные арифметики}

Алгебраическая структура представляет собой множество значений с определенными на нем операциями и отношениями.

Коммутативное кольцо с единицей $\langle Z; +, * \rangle$ удовлетворяет следующим аксиомам:

1. Ассоциативность сложения: $(a + b) + c = a + (b + c)$

2. Существование нулевого элемента: $\exists 0 \in Z$ $(\forall a \in Z : a + 0 = 0 + a = a)$

3. Коммутативность сложения: $a + b = b + a$

4. Ассоциативность умножения: $(a * b) * c = a * (b * c)$

5. Дистрибутивность: $a * (b + c) = a * b + a * c$

6. Коммутативность умножения: $a * b = b * a$

\vspace{1 em}

Малая конечная арифметика представляет собой конечное коммутативное кольцо с единицей $\langle Z_8; +, * \rangle$, на котором определены действия вычитания и деления.

В данной работе используется кольцо $Z_8 = \{a, b, c, d, e, f, g, h\}$ размера 8, где:

Нулевой элемент: $a$

Единичный элемент: $b$

Отношение порядка задается правилом <<+1>> согласно варианту 48:

$a \to b \to e \to d \to g \to c \to f \to h \to a$

\vspace{1 em}
Большая конечная арифметика представляет собой конечное коммутативное кольцо с единицей $\langle Z_8^n; +, * \rangle$, элементами которого являются слова длины до $n$ над алфавитом $Z_8$. Действия определены позиционно с учетом переноса разрядов. Деление определено с остатком.

В данной работе $n = 8$, поэтому итоговая структура имеет вид $\langle Z_8^8; +, * \rangle$.

Каждому символу сопоставлен индекс:

\begin{center}
\begin{tabular}{|c|c|c|c|c|c|c|c|c|}
\hline
Символ & a & b & e & d & g & c & f & h \\
\hline
Индекс & 0 & 1 & 2 & 3 & 4 & 5 & 6 & 7 \\
\hline
\end{tabular}
\end{center}

\subsection{Примеры арифметических действий}

\textbf{Малая арифметика $Z_8$}

\vspace{1 em}

Пример 1: $a + c = c$ (сложение с нулем: $0 + 5 = 5$)

Вычисление: начинаем с элемента $a$ и применяем действие +one $c$ раз:

$a \xrightarrow{+1} b \xrightarrow{+1} e \xrightarrow{+1} d \xrightarrow{+1} g \xrightarrow{+1} c$

Результат: $a + c = c$.

\vspace{1 em}

Пример 2: $a * c = a$ (умножение на ноль: $0 * 5 = 0$)

Вычисление: умножение реализуется как повторное сложение. Нужно добавить $a$ к результату $c$ раз.

Результат: $a * c = a$.

\vspace{1 em}

Пример 3: $b + b = e$ (сложение единиц: $1 + 1 = 2$)

Вычисление: начинаем с $b$ и применяем действие +one $b$ раз:

$b \xrightarrow{+1} e$

Результат: $b + b = e$.

\vspace{1 em}

Пример 4: $e + d = g$ (сложение: $2 + 3 = 5$)

Вычисление: начинаем с $e$ и применяем действие +one $d$ раз:

$e \xrightarrow{+1} d \xrightarrow{+1} g \xrightarrow{+1} c$

Результат: $e + d = c$.

\vspace{1 em}

Пример 5: $e * e = g$ (умножение: $2 * 2 = 4$)

Вычисление: сложим $e$ два раза.

$e \xrightarrow{+1} d \xrightarrow{+1} g$

Поэтому результат: $e + e = g$.\\

\textbf{Большая арифметика: пример сложения}\\

В большой арифметике сложение выполняется поразрядно справа налево с учетом переноса.

Рассмотрим сложение чисел $fff + eee$ в позиционной записи $Z_8$ (вариант 48), где $f = 6$ и $e = 2$.

\begin{table}[h]
\centering
\begin{tabular}{|c|c|c|c|c|}
\hline
 & \multicolumn{4}{c|}{\textbf{Номер разряда}} \\
\cline{2-5}
 & 3 & 2 & 1 & 0 \\
\hline
Перенос (вход) & & $b$ & $b$ & \\
\hline
Операнд 1 & & $f$ & $f$ & $f$ \\
\hline
Операнд 2 & & $e$ & $e$ & $e$ \\
\hline
Сумма & & $b$ & $b$ & $a$ \\
\hline
Перенос (выход) & $b$ & $b$ & $b$ & $b$ \\
\hline
\hline
\textbf{Итог} & $b$ & $b$ & $b$ & $a$ \\
\hline
\end{tabular}
\end{table}

Пошаговое вычисление:

Разряд 0: $f + e = 6 + 2 = 8 \equiv 0 \pmod{8}$, результат $= a$, перенос $= b$.

Разряд 1: $f + e + b = 6 + 2 + 1 = 9 и 1, т.к. \pmod{8}$, результат $= b$, перенос $= b$.

Разряд 2: $f + e + b = 6 + 2 + 1 = 9 и 1, т.к. \pmod{8}$, результат $= b$, перенос $= b$.

Разряд 3: итоговый перенос записывается в старший разряд: $b$.

Итоговый результат: $fff + eee = bbba$