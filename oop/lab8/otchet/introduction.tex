\section{Введение}

В работе рассматривается разработка приложения «Телефонный справочник» с использованием фреймворка Qt. Приложение относится к классу систем управления данными и демонстрирует принципы объектно-ориентированного программирования, работу с пользовательским вводом и хранение информации в файле.

\subsection{Фреймворк Qt}

Qt представляет собой кроссплатформенную библиотеку для создания графических интерфейсов и работы с данными. В работе используются модули Qt Widgets для построения интерфейса, механизм сигналов и слотов для организации взаимодействия между объектами, а также классы QFile и QRegularExpression для работы с файлами и проверки корректности ввода. Дополнительно применяются средства для работы с датами и контейнерами, что позволяет реализовать хранение и обработку контактной информации.

\subsection{Предметная область}

Приложение «Телефонный справочник» предназначено для управления контактной информацией о пользователях. Каждый контакт включает имя, фамилию, отчество, адрес проживания, дату рождения, электронную почту и телефонные номера. Система должна обеспечивать добавление, редактирование и удаление записей, а также их поиск по выбранному столбцу и сортировку также по выбранному столбцу. Для хранения данных используется текстовый файл, а для представления информации пользователю — табличный интерфейс.

