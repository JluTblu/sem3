\section{Результаты работы программы}

\subsection{Запуск калькулятора}

После запуска программы командой \texttt{./calculator} пользователю отображается начальный экран с информацией о системе и доступными операциями (см. рисунок \ref{fig:main_stream}).

\begin{figure}[h!]
    \centering
    \includegraphics[width=0.8\textwidth]{Photo/CleanShot 2025-12-24 at 12.08.52@2x.png}
    \caption{Начальное меню}
    \label{fig:main_stream}
\end{figure}

На начальном экране указаны:

Используемая система счисления и множество элементов.

Значения аддитивной и мультипликативной единиц.

Максимальная длина чисел в разрядах.

Правило инкремента для данного варианта.

Список доступных операций с номерами для выбора.

\subsection{Выполнение операций}

Пользователь вводит номер операции, затем два числа. Программа вычисляет результат и выводит его на экран (см. рисунок \ref{fig:plus}).\\

Пример сложения: $b + e = d$

\newpage

\begin{figure}[h!]
    \centering
    \includegraphics[width=0.8\textwidth]{Photo/CleanShot 2025-12-24 at 12.10.11@2x.png}
    \caption{Действие сложения}
    \label{fig:plus}
\end{figure}

Пользователь выбирает операцию 1, вводит первое число \texttt{b} и второе число \texttt{e}.\\

Пример сложения с отрицательными числами: $-b + f = c$

\begin{figure}[h!]
    \centering
    \includegraphics[width=0.8\textwidth]{Photo/CleanShot 2025-12-24 at 12.14.48@2x.png}
    \caption{Действие сложения с отрицательными значениями}
    \label{fig:plus s minusom}
\end{figure}

Программа корректно обрабатывает отрицательные числа (см. рисунок. \ref{fig:plus s minusom}). Минус указывается перед числом.

Пример вычитания: $f - c = b$

\begin{figure}[h!]
    \centering
    \includegraphics[width=0.8\textwidth]{Photo/CleanShot 2025-12-24 at 12.16.48@2x.png}
    \caption{Действие вычитания}
    \label{fig:minus}
\end{figure}

Вычитание реализовано через сложение с противоположным элементом (см. рисунок. \ref{fig:minus}).\\

Пример умножения: $c * e = be$

\begin{figure}[h!]
    \centering
    \includegraphics[width=0.7\textwidth]{Photo/CleanShot 2025-12-24 at 12.20.47@2x.png}
    \caption{Действие умножения}
    \label{fig:ymnow}
\end{figure}

\newpage

Умножение выполняется через повторное сложение с учетом разрядности результата (см. рисунок. \ref{fig:ymnow}).\\

Пример деления: $h \div c = b(e)$

\begin{figure}[h!]
    \centering
    \includegraphics[width=0.8\textwidth]{Photo/CleanShot 2025-12-24 at 12.27.14@2x.png}
    \caption{Действие деления}
    \label{fig:delenie}
\end{figure}

При делении программа выводит частное и остаток в скобках. Остаток всегда неотрицательный (см. рисунок \ref{fig:delenie}).\\

Пример деления отрицательного числа на положительное: $-h \div c = -e(d)$

\begin{figure}[h!]
    \centering
    \includegraphics[width=0.8\textwidth]{Photo/CleanShot 2025-12-24 at 12.29.24@2x.png}
    \caption{Действие деления отрицательного на положительное}
    \label{fig:delenie2}
\end{figure}

Программа корректно выполняет коррекцию частного и остатка при делении отрицательного числа на положительное (см. рисунок \ref{fig:delenie2}).

\subsubsection{Необычные случаи}

Пример деления аддитивной единицы на аддитивную: $a \div a$

\begin{figure}[h!]
    \centering
    \includegraphics[width=0.8\textwidth]{Photo/CleanShot 2025-12-24 at 12.35.40@2x.png}
    \caption{Обрабатываемый случай деления}
    \label{fig:delenie3}
\end{figure}

Программа корректно выполняет деление и выдает правильный результат при операндах аддитивных единиц (см. рисунок \ref{fig:delenie3}).\\

Пример деления аддитивной единицы на любой другой символ нашей системы счисления: $a \div b$ и $a \div h$

\begin{figure}[h!]
    \centering
    \includegraphics[width=0.7\textwidth]{Photo/CleanShot 2025-12-24 at 12.52.19@2x.png}
    \caption{Случай деления <<a>> на <<x>>}
    \label{fig:delenie4}
\end{figure}

Программа дает корректный результат (см. рисунок \ref{fig:delenie4}), а именно: <<a>>, что и требовалось ожидать.\\

Пример умножения аддитивной единица на любой символ: $a * x = a$

Этот пункт как раз должен выполняться, так как он учитывался в составленном задании к курсовой работе (см. рисунок \ref{fig:ymnow2}).

\begin{figure}[h!]
    \centering
    \includegraphics[width=0.8\textwidth]{Photo/CleanShot 2025-12-24 at 12.59.12@2x.png}
    \caption{Случай умножения <<a>> на <<x>>}
    \label{fig:ymnow2}
\end{figure}

Показанный результат дает понять, что свойство: для любого х | х * a = a

\subsection{Запуск тестов}

Для запуска встроенных тестов пользователь выбирает операцию 5. Результат показан на рисунке \ref{fig:tests}.

Программа последовательно проверяет все математические свойства кольца:

Проверка правила <<+1>> для всех элементов.

Коммутативность сложения.

Ассоциативность сложения.

Коммутативность умножения.

Ассоциативность умножения.

Дистрибутивность умножения относительно сложения.

Свойство умножения на нулевой элемент.

Для каждого теста выводится проверяемое выражение и результат проверки.

\begin{figure}[h!]
    \centering
    \includegraphics[width=0.8\textwidth]{Photo/CleanShot 2025-12-24 at 13.16.01@2x.png}
    \caption{Все тесты}
    \label{fig:tests}
\end{figure}

\subsection{Обработка ошибок}

Программа корректно обрабатывает некорректный ввод и выводит соответствующие сообщения об ошибках.\\

\textbf{Деление на ноль}\\

При попытке деления ненулевого числа на ноль программа выводит сообщение о пустом множестве (см. рисунок \ref{fig:owu6ka1}).

Пример: $c \div a = \{\}$

\newpage

\begin{figure}[h!]
    \centering
    \includegraphics[width=0.8\textwidth]{Photo/CleanShot 2025-12-24 at 13.18.26@2x.png}
    \caption{Деление ненулевого числа на аддитивную единицу}
    \label{fig:owu6ka1}
\end{figure}

\textbf{Переполнение}\\

Если результат операции превышает максимальное количество разрядов, программа выводит сообщение о переполнении (см. рисунок \ref{fig:owu6ka2}).

\begin{figure}[h!]
    \centering
    \includegraphics[width=0.8\textwidth]{Photo/CleanShot 2025-12-24 at 13.26.25@2x.png}
    \caption{Переполнение}
    \label{fig:owu6ka2}
\end{figure}

На скриншоте явно показан пример близкий к переполнению, и что меняется, чтоб мы вышли за пределы 8 символов.

\textbf{Некорректный символ}\\

При вводе символа, не принадлежащего кольцу, программа выводит сообщение об ошибке и заменяет число на нулевое значение (см. рисунок \ref{fig:owu6ka3}).

\begin{figure}[h!]
    \centering
    \includegraphics[width=0.8\textwidth]{Photo/CleanShot 2025-12-24 at 13.30.02@2x.png}
    \caption{Ошибка: некорректный символ}
    \label{fig:owu6ka3}
\end{figure}

\textbf{Некорректный ввод в меню}\\

При вводе некорректного номера действия программа выводит сообщение об ошибке и запрашивает ввод повторно (см. рисунок \ref{fig:owu6ka4}).

\begin{figure}[h!]
    \centering
    \includegraphics[width=0.8\textwidth]{Photo/CleanShot 2025-12-24 at 13.32.37@2x.png}
    \caption{Ошибка: некорректный номер}
    \label{fig:owu6ka4}
\end{figure}

\subsection{Сборка проекта}

Сборка проекта выполняется через Makefile. Команды для сборки и запуска:

\begin{verbatim}
# Сборка проекта
make

# Запуск калькулятора
make run

# Очистка файлов
make clean

# Пересборка
make rebuild
\end{verbatim}

После выполнения команды \texttt{make} создается исполняемый файл \texttt{calculator}. Результат виден на изображении \ref{fig:make}.

\begin{figure}[h!]
    \centering
    \includegraphics[width=0.8\textwidth]{Photo/CleanShot 2025-12-24 at 13.35.55@2x.png}
    \caption{Сборка проекта}
    \label{fig:make}
\end{figure}
 