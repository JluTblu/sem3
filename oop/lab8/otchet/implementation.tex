\section{Реализация}

\subsection{Архитектура приложения}

Приложение построено с использованием объектно-ориентированного подхода и разделения ответственности между компонентами. Структура проекта организована следующим образом: в директории \texttt{headers} располагаются заголовочные файлы классов: Contact.h для структуры данных контакта, ContactDialog.h для диалогового окна редактирования и PhoneBook.h для главного окна приложения. Директория \texttt{src} содержит файлы реализации соответствующих классов, а также файл main.cpp с точкой входа в программу. Файл project.pro описывает конфигурацию проекта для системы сборки qmake.

Архитектура приложения включает несколько ключевых компонентов. Класс Contact представляет собой модель данных для хранения информации об одном контакте. Класс PhoneBook является главным окном приложения и содержит таблицу для отображения всех контактов, а также кнопки управления. Класс ContactDialog реализует диалоговое окно для добавления и редактирования контактов с полной валидацией вводимых данных.

\subsection{Используемые библиотечные классы Qt}

В процессе разработки приложения были использованы базовые классы ядра Qt, классы для работы с данными, а также виджеты графического интерфейса. Ниже приведено краткое описание основных библиотечных компонентов, применённых в реализации.

\subsubsection*{Базовые классы ядра Qt}

\begin{itemize}
    \item \textbf{QString} — класс для работы со строками. Используется для хранения текстовых полей контакта, обработки пользовательского ввода и формирования строк для записи в файл.
    \item \textbf{QList<T>} — контейнерный класс, реализующий список элементов. Применяется для хранения набора объектов \texttt{Contact} и списка телефонных номеров.
    \item \textbf{QDate} — класс для представления и проверки корректности дат. Используется для хранения даты рождения и выполнения валидации.
    \item \textbf{QFile} — класс для работы с файлами. Применяется при чтении и записи данных справочника.
    \item \textbf{QTextStream} — потоковый интерфейс для чтения и записи текстовых данных. Используется совместно с \texttt{QFile} для работы с UTF--8.
    \item \textbf{QRegularExpression} — класс регулярных выражений, применяемый при валидации строковых полей.
\end{itemize}

\subsubsection*{Классы, используемые при наследовании}

\begin{itemize}
    \item \textbf{QWidget} — базовый класс всех визуальных элементов. От него наследуется класс \texttt{PhoneBook}, реализующий главное окно приложения.
    \item \textbf{QDialog} — класс для создания диалоговых окон. От него наследуется \texttt{ContactDialog}, обеспечивающий ввод и редактирование данных контакта.
\end{itemize}

\subsubsection*{Классы графического интерфейса}

\begin{itemize}
    \item \textbf{QTableWidget} — табличный виджет, используемый для отображения списка контактов.
    \item \textbf{QTableWidgetItem} — элемент таблицы, применяемый для заполнения ячеек и хранения пользовательских данных (в частности, индекса контакта).
    \item \textbf{QLineEdit} — однострочное поле ввода. Используется для ввода имени, фамилии, отчества, адреса, email и поиска.
    \item \textbf{QDateEdit} — виджет для выбора даты с поддержкой календаря. Применяется для ввода даты рождения.
    \item \textbf{QPushButton} — кнопки управления (добавление, редактирование, удаление, сохранение, загрузка).
    \item \textbf{QComboBox} — выпадающий список, используемый для выбора поля поиска.
    \item \textbf{QMessageBox} — стандартные диалоговые окна для отображения предупреждений и запросов подтверждения.
    \item \textbf{QFormLayout}, \textbf{QVBoxLayout}, \textbf{QHBoxLayout} — классы компоновки, применяемые для организации интерфейса.
\end{itemize}

Данные классы обеспечивают работу пользовательского интерфейса, обработку данных, взаимодействие с файлом и реализацию механизма валидации.

\subsection{Класс Contact}

\subsubsection{Структура класса Contact}

Класс \texttt{Contact} представляет собой модель данных, описывающую один контакт телефонного справочника. Структура класса включает:\\

\textbf{Поля:}
\begin{itemize}
    \item \texttt{QString lastName} — фамилия;
    \item \texttt{QString firstName} — имя;
    \item \texttt{QString middleName} — отчество;
    \item \texttt{QString address} — адрес проживания;
    \item \texttt{QDate birthDate} — дата рождения;
    \item \texttt{QString email} — адрес электронной почты;
    \item \texttt{QStringList phones} — список телефонных номеров.
\end{itemize}

\textbf{Методы:}
\begin{itemize}
    \item \texttt{toString()} — преобразование объекта в строку для сохранения в файл;
    \item \texttt{fromString()} — восстановление объекта из строки;
    \item геттеры и сеттеры для всех полей.
\end{itemize}

Исходный код класса приведён в Приложении ~\ref{lst:contact-h}.

\subsubsection{Метод toString()}

Метод формирует строковое представление объекта \texttt{Contact} в формате, пригодном для записи в текстовый файл. Поля объединяются фиксированным разделителем, что обеспечивает однозначный разбор данных при загрузке.\\

Алгоритм метода включает:
\begin{itemize}
    \item преобразование даты в формат \texttt{yyyy-MM-dd};
    \item объединение списка телефонов в строку;
    \item формирование итоговой строки с разделителями.
\end{itemize}

Исходный код метода приведён в Приложении ~\ref{lst:contact-cpp}.

\subsubsection{Метод fromString()}

Метод выполняет обратную операцию — преобразует строку из файла в объект \texttt{Contact}.  
Основные этапы:
\begin{itemize}
    \item разбиение строки по разделителю;
    \item проверка корректности количества полей;
    \item преобразование даты из строкового формата;
    \item разбиение строки телефонов на список.
\end{itemize}

Исходный код метода приведён в Приложении ~\ref{lst:contact-cpp}.

\subsubsection{Геттеры и сеттеры}

Геттеры обеспечивают доступ к приватным полям, а сеттеры — контролируемое изменение данных.  
Сеттеры используются также при загрузке данных из файла и при передаче данных между окнами интерфейса.\\

Исходный код геттеров и сеттеров приведён в Приложении ~\ref{lst:contact-cpp}.

\subsubsection{Особенности хранения данных}

Класс \texttt{Contact} обеспечивает:
\begin{itemize}
    \item строгую типизацию полей (например, дата хранится в \texttt{QDate});
    \item независимость внутреннего представления от форматов ввода;
    \item корректное преобразование данных при сохранении и загрузке;
    \item возможность расширения структуры без изменения логики интерфейса.
\end{itemize}

\vspace{1cm}

\subsection{Класс ContactDialog}

\subsubsection{Структура класса ContactDialog}

Класс \texttt{ContactDialog} наследуется от \texttt{QDialog} и реализует диалоговое окно для ввода и редактирования данных контакта.\\

\textbf{Поля:}
\begin{itemize}
    \item \texttt{QLineEdit *lastNameEdit};
    \item \texttt{QLineEdit *firstNameEdit};
    \item \texttt{QLineEdit *middleNameEdit};
    \item \texttt{QLineEdit *addressEdit};
    \item \texttt{QDateEdit *birthDateEdit};
    \item \texttt{QLineEdit *emailEdit};
    \item \texttt{QLineEdit *phonesEdit}.
\end{itemize}

\textbf{Методы:}
\begin{itemize}
    \item \texttt{validateName()} — проверка корректности ФИО;
    \item \texttt{validatePhone()} — проверка телефонного номера;
    \item \texttt{validateEmail()} — проверка адреса электронной почты;
    \item \texttt{validateAndAccept()} — комплексная проверка всех полей и подтверждение ввода;
    \item \texttt{setContact()} — заполнение формы существующими данными;
    \item \texttt{getContact()} — получение объекта \texttt{Contact} из данных формы.
\end{itemize}

Исходный код класса приведён в Приложении ~\ref{lst:dialog-h}.

\subsubsection{Конструктор ContactDialog}

Конструктор класса \texttt{ContactDialog} выполняет инициализацию диалогового окна и настройку всех элементов интерфейса. Логика конструктора включает следующие этапы:

\begin{enumerate}
    \item Создание полей ввода:
    \begin{itemize}
        \item \texttt{QLineEdit} для фамилии, имени, отчества, адреса и email;
        \item \texttt{QDateEdit} для даты рождения с включённым календарём;
        \item \texttt{QLineEdit} для списка телефонных номеров.
    \end{itemize}

    \item Настройка свойств виджетов:
    \begin{itemize}
        \item установка текущей даты рождения по умолчанию;
        \item ограничение максимальной даты рождения текущей датой минус один день;
        \item установка placeholder‑текста для поля телефонов.
    \end{itemize}

    \item Формирование компоновки:
    \begin{itemize}
        \item создание \texttt{QFormLayout} для размещения полей ввода;
        \item создание горизонтального блока кнопок OK/Cancel;
        \item добавление всех элементов в основную компоновку.
    \end{itemize}

    \item Подключение сигналов и слотов:
    \begin{itemize}
        \item сигнал \texttt{clicked()} кнопки OK подключается к слоту \\ \texttt{validateAndAccept()};
        \item сигнал \texttt{clicked()} кнопки Cancel подключается к слоту \\ \texttt{reject()} базового класса \texttt{QDialog}.
    \end{itemize}
\end{enumerate}

Исходный код конструктора приведён в Приложении~\ref{lst:dialog-cpp}.

\subsubsection{Метод validateName()}

Метод проверяет корректность строковых полей ФИО.\\

Проверяются:
\begin{itemize}
    \item отсутствие пустой строки после обрезки пробелов;
    \item корректность первой буквы (заглавная);
    \item корректность регистра остальных символов;
    \item отсутствие дефиса в начале и конце строки;
    \item соответствие регулярному выражению.
\end{itemize}

Исходный код метода приведён в Приложении ~\ref{lst:dialog-cpp}.

\subsubsection{Метод validatePhone()}

Метод проверяет корректность телефонного номера.\\

Проверяются:
\begin{itemize}
    \item допустимые символы (цифры, пробелы, знак <<+>>);
    \item корректность международного или национального формата;
    \item минимальная длина нормализованного номера;
    \item отсутствие недопустимых символов.
\end{itemize}

Исходный код метода приведён в Приложении ~\ref{lst:dialog-cpp}.

\subsubsection{Метод validateEmail()}

Метод проверяет соответствие строки формату \texttt{username@domain.zone}.\\

Проверяются:
\begin{itemize}
    \item наличие символа <<@>>;
    \item корректность имени пользователя;
    \item корректность доменного имени;
    \item наличие точки после домена;
    \item корректность доменной зоны.
\end{itemize}

Исходный код метода приведён в Приложении ~\ref{lst:dialog-cpp}.

\subsubsection{Метод validateAndAccept()}

Метод выполняет комплексную проверку всех полей формы.\\

Алгоритм:
\begin{itemize}
    \item последовательный вызов методов валидации;
    \item вывод сообщения об ошибке при первой некорректности;
    \item создание объекта \texttt{Contact} при успешной проверке;
    \item закрытие диалога с результатом \texttt{Accepted}.
\end{itemize}

Исходный код метода приведён в Приложении ~\ref{lst:dialog-cpp}.

\subsection{Класс PhoneBook}

Класс \texttt{PhoneBook} представляет главное окно приложения и наследуется от \texttt{QWidget}. Он отвечает за отображение списка контактов, обработку пользовательских действий и взаимодействие с файловой системой.

\subsubsection{Структура класса PhoneBook}

\textbf{Поля:}
\begin{itemize}
    \item \texttt{QTableWidget *table} — таблица для отображения контактов;
    \item \texttt{QLineEdit *searchEdit} — поле ввода поискового запроса;
    \item \texttt{QList<Contact> contacts} — внутренний список всех контактов;
    \item \texttt{QString filename} — путь к файлу данных.
\end{itemize}

\textbf{Методы:}
\begin{itemize}
    \item \texttt{updateTable()} — обновление содержимого таблицы;
    \item \texttt{addContact()} — добавление нового контакта;
    \item \texttt{editContact()} — редактирование выбранного контакта;
    \item \texttt{deleteContact()} — удаление выбранного контакта;
    \item \texttt{searchByColumn(int)} — поиск по выбранному столбцу;
    \item \texttt{saveToFile()} — сохранение данных в файл;
    \item \texttt{loadFromFile()} — загрузка данных из файла.
\end{itemize}

Исходный код класса приведён в Приложении~\ref{lst:phonebook-h}.

\subsubsection{Конструктор PhoneBook(QWidget *parent)}

Конструктор класса \texttt{PhoneBook} выполняет инициализацию главного окна приложения, создание интерфейса и настройку всех элементов. Логика конструктора включает следующие этапы:

\begin{enumerate}
    \item Настройка окна:
    \begin{itemize}
        \item установка заголовка окна;
        \item установка минимального размера;
        \item определение пути к файлу данных.
    \end{itemize}

    \item Создание панели поиска:
    \begin{itemize}
        \item создание поля ввода \texttt{QLineEdit};
        \item создание выпадающего списка \texttt{QComboBox} для выбора столбца поиска;
        \item заполнение списка названиями столбцов.
    \end{itemize}

    \item Создание таблицы контактов:
    \begin{itemize}
        \item создание \texttt{QTableWidget} с семью столбцами;
        \item установка заголовков столбцов;
        \item включение автоматической сортировки;
        \item настройка режима выбора строк.
    \end{itemize}

    \item Создание панели кнопок:
    \begin{itemize}
        \item создание кнопок «Добавить», «Редактировать», «Удалить», «Сохранить», «Загрузить».
    \end{itemize}

    \item Подключение сигналов и слотов:
    \begin{itemize}
        \item сигнал \texttt{clicked()} кнопки «Добавить» -> слот \texttt{addContact()};
        \item сигнал \texttt{clicked()} кнопки «Редактировать» -> слот \texttt{editContact()};
        \item сигнал \texttt{clicked()} кнопки «Удалить» -> слот \texttt{deleteContact()};
        \item сигнал \texttt{clicked()} кнопки «Сохранить» -> слот \texttt{saveToFile()};
        \item сигнал \texttt{clicked()} кнопки «Загрузить» -> слот \texttt{loadFromFile()};
        \item сигнал изменения текста в поле поиска \texttt{textChanged()} вызывает \\ лямбда‑функцию, которая передаёт выбранный столбец \\ в слот \texttt{searchByColumn(int)}.
    \end{itemize}

    \item Формирование основной компоновки:
    \begin{itemize}
        \item размещение панели поиска;
        \item размещение таблицы;
        \item размещение панели кнопок;
        \item установка компоновки в главное окно.
    \end{itemize}
\end{enumerate}

Исходный код конструктора приведён в Приложении~\ref{lst:phonebook-cpp}.

\subsubsection{Метод updateTable()}

Метод \texttt{updateTable()} синхронизирует визуальное представление таблицы с внутренним списком контактов.\\

\newpage

\begin{figure}[h!]
\centering
\includegraphics[width=0.9\textwidth]{Photo/CleanShot 2026-01-11 at 03.42.46.png}
\caption{Блок‑схема алгоритма обновления таблицы контактов}
\label{fig:updateTable}
\end{figure}

Алгоритм работы включает:

\begin{enumerate}
    \item временное отключение сортировки таблицы;
    \item изменение количества строк в соответствии с размером списка;
    \item для каждой строки:
    \begin{itemize}
        \item получение объекта \texttt{Contact};
        \item создание элементов таблицы для каждого поля;
        \item запись индекса контакта в \texttt{UserRole} первой ячейки;
        \item заполнение остальных ячеек значениями полей;
    \end{itemize}
    \item повторное включение сортировки.
\end{enumerate}

Исходный код метода приведён в Приложении~\ref{lst:phonebook-cpp}.

\subsubsection{Метод addContact()}

Метод \texttt{addContact()} отвечает за добавление нового контакта.

\newpage

\begin{figure}[h!]
\centering
\includegraphics[width=0.7\textwidth]{Photo/CleanShot 2026-01-11 at 03.50.34.png}
\caption{Блок‑схема алгоритма добавления нового контакта}
\label{fig:addContact}
\end{figure}

Логика работы:

\begin{enumerate}
    \item Создать диалоговое окно \texttt{ContactDialog}.
    \item Открыть диалог и дождаться результата.
    \item Если пользователь подтвердил ввод:
    \begin{itemize}
        \item получить объект \texttt{Contact};
        \item добавить его в список \texttt{contacts};
        \item обновить таблицу вызовом \texttt{updateTable()}.
    \end{itemize}
\end{enumerate}

Исходный код метода приведён в Приложении~\ref{lst:phonebook-cpp}.

\subsubsection{Метод editContact()}

Метод \texttt{editContact()} выполняет редактирование выбранного контакта.\\

Алгоритм:

\begin{enumerate}
    \item Получить номер выбранной строки.
    \item Если строка не выбрана — вывести предупреждение.
    \item Извлечь реальный индекс контакта из \texttt{UserRole}.
    \item Создать диалог \texttt{ContactDialog} и заполнить его текущими данными.
    \item Если пользователь подтвердил изменения:
    \begin{itemize}
        \item обновить объект в списке;
        \item вызвать \texttt{updateTable()}.
    \end{itemize}
\end{enumerate}

Исходный код метода приведён в Приложении~\ref{lst:phonebook-cpp}.

\subsubsection{Метод deleteContact()}

Метод \texttt{deleteContact()} удаляет выбранный контакт.\\

Алгоритм:

\begin{enumerate}
    \item Проверить, выбрана ли строка.
    \item Если нет — вывести предупреждение.
    \item Запросить подтверждение удаления через \texttt{QMessageBox}.
    \item Если пользователь подтвердил:
    \begin{itemize}
        \item извлечь индекс контакта из \texttt{UserRole};
        \item удалить элемент из списка \texttt{contacts};
        \item обновить таблицу.
    \end{itemize}
\end{enumerate}

Исходный код метода приведён в Приложении~\ref{lst:phonebook-cpp}.

\subsubsection{Метод searchByColumn(int column)}

Метод \texttt{searchByColumn()} выполняет фильтрацию строк таблицы.\\

Алгоритм:

\newpage

\begin{figure}[h!]
    \centering
    \includegraphics[width=0.7\textwidth]{Photo/CleanShot 2026-01-11 at 13.02.08.png}
    \caption{Блок‑схема алгоритма поиска по выбранному столбцу}
    \label{fig:search}
\end{figure}

Исходный код метода приведён в Приложении~\ref{lst:phonebook-cpp}.

\subsubsection{Метод saveToFile()}

Метод \texttt{saveToFile()} выполняет сохранение данных в текстовый файл.\\

Алгоритм:

\begin{enumerate}
    \item Открыть файл в режиме записи.
    \item Создать поток \texttt{QTextStream} с кодировкой UTF--8.
    \item Для каждого контакта:
    \begin{itemize}
        \item получить строковое представление через \texttt{toString()};
        \item записать строку в файл.
    \end{itemize}
    \item Закрыть файл.
    \item Вывести сообщение об успешном сохранении.
\end{enumerate}

Исходный код метода приведён в Приложении~\ref{lst:phonebook-cpp}.

\subsubsection{Метод loadFromFile()}

Метод \texttt{loadFromFile()} загружает данные из файла.\\

Алгоритм:

\begin{enumerate}
    \item Открыть файл в режиме чтения.
    \item Очистить текущий список контактов.
    \item Читать файл построчно до конца.
    \item Для каждой строки:
    \begin{itemize}
        \item создать объект \texttt{Contact} через \texttt{fromString()};
        \item если данные корректны — добавить в список.
    \end{itemize}
    \item Закрыть файл.
    \item Обновить таблицу.
\end{enumerate}

Исходный код метода приведён в Приложении~\ref{lst:phonebook-cpp}.

\subsubsection{Механизм сигналов и слотов}

Класс \texttt{PhoneBook} использует механизм сигналов и слотов Qt для обработки действий пользователя. В конструкторе выполняются следующие подключения:

\begin{itemize}
    \item кнопка «Добавить» -> слот \texttt{addContact()};
    \item кнопка «Редактировать» -> слот \texttt{editContact()};
    \item кнопка «Удалить» -> слот \texttt{deleteContact()};
    \item кнопка «Сохранить» -> слот \texttt{saveToFile()};
    \item кнопка «Загрузить» -> слот \texttt{loadFromFile()};
    \item изменение текста в поле поиска вызывает лямбда‑функцию, передающую выбранный столбец в \texttt{searchByColumn(int)}.
\end{itemize}

Исходный код подключений приведён в Приложении~\ref{lst:phonebook-cpp}.

\subsection{Регулярные выражения, используемые в приложении}

Валидация пользовательского ввода в приложении основана на использовании регулярных выражений. В данном разделе приведён общий синтаксис регулярных выражений, применяемый в Qt, а также частные выражения, использованные при проверке полей контакта.

\subsubsection{Общий синтаксис регулярных выражений}

Регулярные выражения позволяют описывать шаблоны строк с использованием специальных конструкций. Наиболее важные элементы синтаксиса:

\begin{itemize}
    \item \textbf{Квантификаторы:}
    \begin{itemize}
        \item \texttt{+} — один или более повторений;
        \item \texttt{*} — ноль или более повторений;
        \item \texttt{?} — ноль или одно повторение;
        \item \texttt{\{n,m\}} — от \texttt{n} до \texttt{m} повторений.
    \end{itemize}

    \item \textbf{Якоря:}
    \begin{itemize}
        \item \texttt{\^} — начало строки;
        \item \texttt{\$} — конец строки.
    \end{itemize}

    \item \textbf{Символьные классы:}
    \begin{itemize}
        \item \texttt{[abc]} — любой символ из набора;
        \item \texttt{[a-z]} — диапазон символов;
        \item \texttt{\textbackslash d} — цифра;
        \item \texttt{\textbackslash s} — пробельный символ;
        \item \texttt{\textbackslash p\{L\}} — любая буква любого алфавита.
    \end{itemize}

    \item \textbf{Группировка:}
    \begin{itemize}
        \item \texttt{( ... )} — логическая группа символов.
    \end{itemize}
\end{itemize}

Qt использует класс \texttt{QRegularExpression}.

\subsubsection{Регулярные выражения, применённые в курсовой работе}

Ниже приведены частные регулярные выражения, использованные для проверки корректности полей контакта.

\paragraph{1. Проверка имени, фамилии и отчества}

\begin{verbatim}
^[\p{L}\d\s-]+$
\end{verbatim}

\textbf{Пояснение:}
\begin{itemize}
    \item \texttt{\^} и \texttt{\$} — строка должна полностью соответствовать шаблону;
    \item \texttt{\textbackslash p\{L\}} — буквы любых алфавитов;
    \item \texttt{\textbackslash d} — допускаются цифры;
    \item \texttt{\textbackslash s} — пробелы внутри строки;
    \item \texttt{-} — разрешён дефис;
    \item \texttt{+} — один или более символов.
\end{itemize}

Это выражение используется после проверки регистра первой буквы и запрета дефиса в начале / конце строки.

\paragraph{2. Проверка телефонного номера}

\begin{verbatim}
^\+?\d[\d\s]*$
\end{verbatim}

\textbf{Пояснение:}
\begin{itemize}
    \item \texttt{\^} — начало строки;
    \item \texttt{\+?} — необязательный символ «+»;
    \item \texttt{\d} — номер должен начинаться с цифры или «+цифра»;
    \item \texttt{[\d\s]*} — цифры и пробелы в любом количестве;
    \item \texttt{\$} — конец строки.
\end{itemize}

После проверки строка нормализуется до формата «+цифры».

\paragraph{3. Проверка адреса электронной почты}

\begin{verbatim}
^[A-Za-z0-9]+@[A-Za-z0-9]+\.[A-Za-z0-9]+$
\end{verbatim}

\textbf{Пояснение:}
\begin{itemize}
    \item \texttt{[A-Za-z0-9]+} — имя пользователя: буквы и цифры;
    \item \texttt{@} — обязательный разделитель;
    \item \texttt{[A-Za-z0-9]+} — доменное имя;
    \item \texttt{\.} — точка между доменом и зоной;
    \item \texttt{[A-Za-z0-9]+} — доменная зона.
\end{itemize}

Пробелы предварительно удаляются автоматически.\\

\noindent\textbf{4. Проверка даты рождения}\\

Дата проверяется не регулярным выражением, а средствами класса \texttt{QDate}, однако валидация включает:

\begin{itemize}
    \item корректность календарной даты;
    \item проверку, что дата строго меньше текущей.
\end{itemize}

Таким образом, регулярные выражения используются только для строковых полей.

