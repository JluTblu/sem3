\section{Тестирование приложения}

\subsection{Начальное состояние приложения}

При первом запуске приложения отображается главное окно с пустой таблицей контактов (см. рисунок \ref{fig:firstload}), полем поиска в верхней части и кнопками управления в нижней части.

Главное окно имеет минимальный размер 800x600 пикселей. Таблица настроена на автоматическое растягивание последнего столбца для заполнения всей ширины окна. Режим выбора установлен на SelectRows, что позволяет выделять только целые строки.

\begin{figure}[h!]
    \centering
    \includegraphics[width=0.9\textwidth]{Photo/CleanShot 2025-12-13 at 19.13.10@2x.png}
    \caption{Главное окно при первом запуске}
    \label{fig:firstload}
\end{figure}

\subsection{Добавление нового контакта}

Пользователь нажимает кнопку «Добавить», после чего открывается диалоговое окно с формой ввода (см. рисунок \ref{fig:add}). Пользователь заполняет все обязательные поля: фамилию «Иванов», имя «Иван», отчество «Иванович», адрес «Москва, ул. Ленина, д. 1», дату рождения через всплывающий календарь, email \\ «ivanov@example.com» и телефоны «+79991234567, +74951234567».

После нажатия кнопки OK система выполняет валидацию всех полей. Если все данные корректны, диалоговое окно закрывается, новый контакт добавляется в список, и таблица обновляется. Пользователь видит новую строку с введенными данными. Данные автоматически сохраняются в файл phonebook.txt.

\begin{figure}[h!]
    \centering
    \includegraphics[width=0.9\textwidth]{Photo/CleanShot 2025-12-13 at 19.26.55@2x.png}
    \caption{Добавление контакта}
    \label{fig:add}
\end{figure}

\subsection{Валидация некорректных данных}

При попытке добавить контакт с некорректными данными система выводит соответствующие сообщения об ошибках. \\

\subsubsection{Некорректное имя} \\ Если пользователь вводит имя с маленькой буквы, например «иван» (см. рисунок \ref{fig:name}), система выводит сообщение «Имя должно начинаться с заглавной буквы». Диалоговое окно остается открытым, позволяя пользователю исправить ошибку.

\newpage

\begin{figure}[h!]
    \centering
    \includegraphics[width=0.7\textwidth]{Photo/CleanShot 2025-12-13 at 19.31.53@2x.png}
    \caption{Ввод некорректного имени}
    \label{fig:name}
\end{figure}

\subsubsection{Некорректный email} \\ Если пользователь вводит email без символа @ или без доменной зоны, например «ivanovexample.com», как на рисунке \ref{fig:mail}, или «ivanov@example», система выводит сообщение «Неверный формат email». Пользователь должен исправить формат адреса перед сохранением.

\begin{figure}[h!]
    \centering
    \includegraphics[width=0.7\textwidth]{Photo/CleanShot 2025-12-13 at 19.34.59@2x.png}
    \caption{Ввод некорректной почты}
    \label{fig:mail}
\end{figure}

\subsubsection{Некорректный телефон} \\ Если пользователь вводит слишком короткий номер телефона, например «123» (см. рисунок \ref{fig:phone}), система выводит сообщение «Телефон слишком короткий: 123». Минимальная длина нормализованного номера должна составлять 10 символов.

\begin{figure}[h!]
    \centering
    \includegraphics[width=0.7\textwidth]{Photo/CleanShot 2025-12-13 at 19.37.35@2x.png}
    \caption{Ввод некорректного мобильного телефона}
    \label{fig:phone}
\end{figure}

Также учитывается случай, когда пользователь в номер телефона вводит символы, пусть латиницей. Они при записи просто перестают учитываться и не записываются в поле контакта (см. рисунок \ref{fig:phone2}).

\begin{figure}[h!]
    \centering
    \includegraphics[width=0.7\textwidth]{Photo/CleanShot 2025-12-13 at 19.41.42@2x.png}
    \caption{Ввод символов в поле <<Телефоны>>}
    \label{fig:phone2}
\end{figure}

\subsubsection{fieldName с дефисом в начале} \\ Если пользователь вводит фамилию (имя, отчество), начинающуюся с дефиса, например «-Иванов», система выводит сообщение «Фамилия должна начинаться с заглавной буквы».

\begin{figure}[h!]
    \centering
    \includegraphics[width=0.7\textwidth]{Photo/CleanShot 2025-12-13 at 19.45.04@2x.png}
    \caption{Пример некорректного заполнения поля <<Фамилия>>}
    \label{fig:surname}
\end{figure}

\subsection{Редактирование существующего контакта}

Пользователь выбирает строку в таблице, кликая по ней. После выбора становится активной кнопка «Редактировать». При нажатии на эту кнопку открывается диалоговое окно ContactDialog, все поля которого уже заполнены данными выбранного контакта.

Пользователь изменяет необходимые поля, например, обновляет адрес на «Санкт-Петербург, Невский проспект, д. 10» и добавляет еще один телефон. После нажатия кнопки OK система выполняет валидацию измененных данных. Если все данные корректны, обновленный контакт заменяет старую запись в списке, таблица обновляется, показывая новые данные. Все действия показаны на скриншотах \ref{fig:process} и \ref{fig:res}.

\newpage

\begin{figure}[h!]
    \centering
    \includegraphics[width=0.6\textwidth]{Photo/CleanShot 2025-12-13 at 19.50.24@2x.png}
    \caption{Процесс редактирования}
    \label{fig:process}
\end{figure}

\begin{figure}[h!]
    \centering
    \includegraphics[width=0.6\textwidth]{Photo/CleanShot 2025-12-13 at 19.51.14@2x.png}
    \caption{Результат редактирования}
    \label{fig:res}
\end{figure}

\subsection{Удаление контакта}

Пользователь выбирает строку в таблице, после чего становится активной кнопка «Удалить». При нажатии на кнопку система выводит диалоговое окно с вопросом «Удалить выбранный контакт?» и кнопками «Yes» и «No», как показано на рисунке \ref{fig:delete}. Также программа учитывает, что всегда выделен какой-либо контакт пользователем, т.е. множественное удаление не может происходить - так как отключено множественное выделение:

\begin{verbatim}
    table->setSelectionMode(QAbstractItemView::SingleSelection);
\end{verbatim}

\begin{figure}[h!]
    \centering
    \includegraphics[width=0.7\textwidth]{Photo/CleanShot 2025-12-13 at 19.57.50@2x.png}
    \caption{Удаление выбранного контакта}
    \label{fig:delete}
\end{figure}

Если пользователь подтверждает удаление, нажимая «Yes», контакт удаляется из списка contacts, строка исчезает из таблицы. Если пользователь нажимает «No», операция отменяется, и контакт остается в справочнике.

\subsection{Поиск}

\subsubsection{Поиск по фамилии}\\ Пользователь вводит текст «Иванов» в поле поиска. Система в реальном времени фильтрует таблицу, оставляя видимыми только те строки, которые содержат введенную подстроку в любом из полей. Все остальные строки скрываются.

\begin{figure}[h!]
    \centering
    \includegraphics[width=0.7\textwidth]{Photo/CleanShot 2025-12-13 at 20.03.41@2x.png}
    \caption{Поиск по фамилии}
    \label{fig:search1}
\end{figure}

\subsubsection{Поиск по email}\\  Пользователь вводит часть email, например «@example». Система показывает только те контакты, у которых в поле email присутствует данная подстрока. Поиск не чувствителен к регистру символов.

\begin{figure}[h!]
    \centering
    \includegraphics[width=0.7\textwidth]{Photo/CleanShot 2025-12-13 at 20.07.04@2x.png}
    \caption{Поиск по почте}
    \label{fig:search2}
\end{figure}

\subsubsection{Поиск по телефону}\\ Пользователь вводит часть телефонного номера, например «999». Система показывает все контакты, у которых хотя бы один телефон содержит эту последовательность цифр.

\begin{figure}[h!]
    \centering
    \includegraphics[width=0.7\textwidth]{Photo/CleanShot 2025-12-13 at 20.08.08@2x.png}
    \caption{Поиск по телефону}
    \label{fig:search3}
\end{figure}

Рассмотрим ситуацию: когда у контакта два телефона, так как они записываются через запятую, то поиск не учитывает эту запятую, а рассматривает только цифры в номере (см. рисунок \ref{fig:search4}). Одна из микроособенностей данной программы.
\begin{figure}[h!]
    \centering
    \includegraphics[width=0.7\textwidth]{Photo/CleanShot 2025-12-13 at 20.10.48@2x.png}
    \caption{Проверка на поиск при вводе сразу двух номеров}
    \label{fig:search4}
\end{figure}

\subsubsection{Очистка поиска}\\ Когда пользователь очищает поле поиска, все строки таблицы снова становятся видимыми, показывая полный список контактов.

\begin{figure}[h!]
    \centering
    \includegraphics[width=0.7\textwidth]{Photo/CleanShot 2025-12-13 at 20.13.01@2x.png}
    \caption{Пустой поиск}
    \label{fig:search5}
\end{figure}

\subsection{Сортировка данных}

\subsubsection{Сортировка по фамилии}\\ Пользователь кликает на заголовок столбца «Фамилия». Система автоматически сортирует все строки таблицы по фамилии в алфавитном порядке по возрастанию (см. рисунок \ref{fig:sort1}). При повторном клике на тот же заголовок направление сортировки меняется на убывание.

\begin{figure}[h!]
    \centering
    \includegraphics[width=0.7\textwidth]{Photo/CleanShot 2025-12-13 at 20.14.32@2x.png}
    \caption{Сортировка в алфавитном порядке}
    \label{fig:sort1}
\end{figure}

\subsubsection{Сортировка по дате рождения}\\ Пользователь кликает на заголовок столбца «Дата рождения». Система сортирует контакты по датам, располагая их от самых старых к самым молодым (см. рисунок \ref{fig:sort2}). Повторный клик меняет порядок на обратный.

\begin{figure}[h!]
    \centering
    \includegraphics[width=0.7\textwidth]{Photo/CleanShot 2025-12-13 at 20.16.51@2x.png}
    \caption{Сортировка по дате}
    \label{fig:sort2}
\end{figure}

\subsubsection{Сортировка по email} \\ Система поддерживает сортировку по любому столбцу таблицы. Клик по заголовку столбца «Email» сортирует контакты в алфавитном порядке по адресам электронной почты (см. рисунок \ref{fig:sort3}).

\begin{figure}[h!]
    \centering
    \includegraphics[width=0.7\textwidth]{Photo/CleanShot 2025-12-13 at 20.18.31@2x.png}
    \caption{Сортировка в алфавитном порядке по полю <<Email>>}
    \label{fig:sort3}
\end{figure}

Механизм сортировки реализован встроенными средствами QTableWidget через флаг setSortingEnabled(true), что обеспечивает автоматическую обработку кликов по заголовкам столбцов.

\subsection{Работа с файлами}

\subsubsection{Автоматическая загрузка при запуске}\\ При запуске приложения конструктор PhoneBook вызывает метод loadFromFile, который пытается открыть файл phonebook.txt. Если файл существует и содержит корректные данные, все контакты загружаются в память и отображаются в таблице. Если файл не существует, приложение запускается с пустым справочником.

\subsubsection{Автоматическое сохранение после изменений}\\ После каждой операции добавления, редактирования или удаления контакта метод updateTable вызывает saveToFile, который записывает текущее состояние списка контактов в файл. Это гарантирует, что данные не будут потеряны при внезапном закрытии приложения.

\subsubsection{Ручное сохранение} \\ Пользователь может явно вызвать сохранение данных, нажав кнопку «Сохранить». Если операция выполнена успешно, система выводит сообщение «Данные сохранены» (см. рисунок \ref{fig:save}). Если возникла ошибка при открытии файла, отображается предупреждение «Не удалось открыть файл для записи».

\begin{figure}[h!]
    \centering
    \includegraphics[width=0.7\textwidth]{Photo/CleanShot 2025-12-13 at 20.20.50@2x.png}
    \caption{Ручное сохранение в файл}
    \label{fig:save}
\end{figure}

\subsubsection{Ручная загрузка} \\ Кнопка «Загрузить» позволяет перезагрузить данные из файла. Это полезно, если файл был изменен внешним приложением. Текущее содержимое списка contacts полностью заменяется данными из файла, после чего таблица обновляется.

Сами же данные в файле записаны следующим образом:

\begin{figure}[h!]
    \centering
    \includegraphics[width=0.9\textwidth]{Photo/CleanShot 2025-12-13 at 20.22.56@2x.png}
    \caption{phonebook.txt}
    \label{fig:txt}
\end{figure}

\subsection{Использование календаря для выбора даты}

При добавлении или редактировании контакта пользователь может выбрать дату рождения с помощью виджета QDateEdit. Справа от поля даты располагается кнопка с иконкой календаря. При клике на эту кнопку появляется всплывающий календарь QCalendarWidget.

Пользователь может ориентироваться по месяцам и годам, используя стрелки в верхней части календаря. Клик по конкретной дате закрывает календарь и устанавливает выбранное значение в поле даты. Виджет настроен так, что максимально допустимая дата --- это вчерашний день, предотвращая выбор будущих дат или текущего дня. Пример приведен на рисунке \ref{fig:calendar}.

\begin{figure}[h!]
    \centering
    \includegraphics[width=0.7\textwidth]{Photo/CleanShot 2025-12-13 at 20.26.08@2x.png}
    \caption{Виджет-календарь}
    \label{fig:calendar}
\end{figure}

\subsection{Тестирование регулярных выражений}

Для проверки корректности работы валидации был проведен ряд тестов с различными входными значениями.

\subsubsection{Тестирование валидации имени} \\ Имя «Иван» проходит валидацию успешно. Имя «иван» отклоняется с сообщением о необходимости заглавной буквы. Имя «Иван-Петр» проходит валидацию, так как дефис допустим внутри строки. Имя «-Иван» отклоняется, так как не может начинаться с дефиса. Имя «Иван123» проходит валидацию, поскольку цифры разрешены.

\subsubsection{Тестирование валидации email} \\ Адрес «test@mail.ru» проходит валидацию успешно. Адрес «test@mail» отклоняется из-за отсутствия доменной зоны. Адрес «testmail.ru» отклоняется из-за отсутствия символа @. Адрес «test @mail.ru» корректируется автоматически путем удаления всех пробелов перед проверкой.

\subsubsection{Тестирование валидации телефона} \\ Номер «+79991234567» проходит валидацию и нормализуется до «+79991234567». Номер «8(999)123-45-67» проходит валидацию и нормализуется до «89991234567». Номер «123» отклоняется как слишком короткий. Номер «+7 999 123 45 67» проходит валидацию и нормализуется до «+79991234567». 

\subsubsection{Тестирование валидации даты} \\ Дата 15.03.2000 проходит валидацию успешно. Текущая дата или любая будущая дата отклоняется системой, так как виджет QDateEdit настроен с максимальным значением на вчерашний день.

\subsection{Результаты тестирования}

В результате тестирования были проверены следующие аспекты функциональности приложения.

\subsubsection{Функциональность операций} \\ Все операции добавления, чтения, обновления и удаления контактов работают корректно. Новые контакты успешно добавляются в список и отображаются в таблице. Редактирование позволяет изменять любые поля существующего контакта. Удаление корректно удаляет выбранный контакт после подтверждения пользователя.

\subsubsection{Валидация данных} \\ Система эффективно отсекает некорректные данные на этапе ввода. Пользователь получает понятные сообщения об ошибках с указанием конкретного поля и характера проблемы. Регулярные выражения правильно проверяют формат имен, email и телефонных номеров.

\subsubsection{Поиск и фильтрация} \\ Система корректно фильтрует данные в реальном времени по мере ввода поискового запроса. Поиск работает по всем полям таблицы одновременно. Реализация через скрытие строк обеспечивает мгновенную реакцию на действия пользователя.

\subsubsection{Сортировка} \\ Механизм сортировки работает корректно для всех столбцов таблицы. Пользователь может сортировать данные как по текстовым полям, так и по дате. Смена направления сортировки при повторном клике работает ожидаемым образом.

\subsubsection{Работа с файлами} \\ Сохранение и загрузка данных происходят без потери информации. Формат хранения с использованием разделителей обеспечивает простоту и читаемость файла.

\subsubsection{Интерфейс} \\ Все элементы интерфейса корректно реагируют на действия пользователя. Состояние кнопок меняется в зависимости от контекста. Диалоговые окна правильно открываются и закрываются. Таблица адекватно отображает данные и поддерживает интерактивное взаимодействие.