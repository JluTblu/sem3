\section{Тестирование приложения}

В данном разделе приведено описание стартового окна приложения, доступных пользователю элементов интерфейса, а также сценарии тестирования в формате use‑case. Для каждой функциональной задачи представлены корректные и некорректные варианты взаимодействия.

\subsection{Описание стартового окна}

После запуска приложения пользователь видит главное окно \texttt{PhoneBook}. Интерфейс включает следующие элементы:

\begin{itemize}
    \item \textbf{Панель поиска}:
    \begin{itemize}
        \item поле ввода поискового текста (\texttt{QLineEdit});
        \item выпадающий список выбора столбца (\texttt{QComboBox});
        \item оба элемента активны сразу после запуска.
    \end{itemize}

    \item \textbf{Таблица контактов} (\texttt{QTableWidget}):
    \begin{itemize}
        \item отображает список контактов (при первом запуске — пустая);
        \item сортировка по столбцам активна;
        \item режим выбора — по строкам;
        \item редактирование ячеек пользователем отключено.
    \end{itemize}

    \item \textbf{Панель кнопок управления}:
    \begin{itemize}
        \item «Добавить» — активна;
        \item «Редактировать» — активна, но требует выбранной строки;
        \item «Удалить» — активна, но требует выбранной строки;
        \item «Сохранить» — активна всегда;
        \item «Загрузить» — активна всегда.
    \end{itemize}
\end{itemize}

\newpage

\begin{figure}[h!]
\centering
\includegraphics[width=0.8\textwidth]{Photo/CleanShot 2025-12-13 at 19.13.10@2x.png}
\caption{Стартовое окно приложения: панель поиска, таблица контактов и панель управления}
\label{fig:start-window}
\end{figure}

Таким образом, тестировщик сразу видит доступные действия и может проверить корректность поведения интерфейса при различных сценариях.

\subsection{Use‑case сценарии тестирования}

Ниже приведены сценарии тестирования основных функций приложения, соответствующих задачам, описанным в разделе ~\ref{sec:task}.

\subsubsection{Use‑case 1: Добавление нового контакта}

\textbf{Цель:} пользователь добавляет корректный контакт в справочник.

\newpage

\begin{figure}[h!]
\centering
\includegraphics[width=0.8\textwidth]{Photo/CleanShot 2025-12-13 at 19.26.55@2x.png}
\caption{Диалоговое окно добавления нового контакта}
\label{fig:add-dialog}
\end{figure}

\textbf{Основной сценарий (корректный):}
\begin{enumerate}
    \item Пользователь нажимает кнопку «Добавить».
    \item Открывается диалоговое окно ввода данных.
    \item Пользователь заполняет все поля корректными значениями:
    \begin{itemize}
        \item ФИО — корректный формат;
        \item дата рождения — меньше текущей;
        \item email — соответствует шаблону;
        \item телефоны — корректные номера.
    \end{itemize}
    \item Пользователь нажимает «ОК».
    \item Контакт появляется в таблице.
\end{enumerate}

\begin{figure}[h!]
\centering
\includegraphics[width=0.8\textwidth]{Photo/CleanShot 2025-12-13 at 19.51.14@2x.png}
\caption{Таблица после успешного добавления контакта}
\label{fig:contact-added}
\end{figure}

\newpage

\textbf{Альтернативные сценарии (некорректные):}
\begin{itemize}
    \item Поле ФИО пустое (см. рисунок \ref{fig:contact-added1}) -> выводится предупреждение, диалог не закрывается.
    \item ФИО начинается с маленькой буквы или недопустимого символа <<->> (см. рисунки \ref{fig:contact-added2}) и \ref{fig:contact-added7}) -> предупреждение.
    \item Email без символа «@» (см. рисунок \ref{fig:contact-added3}) -> предупреждение.
    \item Телефон содержит недопустимые символы (см. рисунки \ref{fig:contact-added4} и \ref{fig:contact-added6}) -> предупреждение.
    \item Дата рождения больше текущей (см. рисунок \ref{fig:contact-added5}) -> предупреждение.
\end{itemize}

\begin{figure}[h!]
\centering
\includegraphics[width=0.5\textwidth]{Photo/CleanShot 2026-01-11 at 13.31.23.png}
\caption{Случай некорректной записи}
\label{fig:contact-added1}
\end{figure}

\newpage

\begin{figure}[h!]
\centering
\includegraphics[width=0.7\textwidth]{Photo/CleanShot 2025-12-13 at 19.31.53@2x.png}
\caption{Случай некорректной записи}
\label{fig:contact-added2}
\end{figure}

\begin{figure}[h!]
\centering
\includegraphics[width=0.7\textwidth]{Photo/CleanShot 2025-12-13 at 19.45.04@2x.png}
\caption{Случай некорректной записи}
\label{fig:contact-added7}
\end{figure}

\newpage

\begin{figure}[h!]
\centering
\includegraphics[width=0.7\textwidth]{Photo/CleanShot 2025-12-13 at 19.34.59@2x.png}
\caption{Случай некорректной записи}
\label{fig:contact-added3}
\end{figure}

\begin{figure}[h!]
\centering
\includegraphics[width=0.7\textwidth]{Photo/CleanShot 2025-12-13 at 19.37.35@2x.png}
\caption{Случай некорректной записи}
\label{fig:contact-added4}
\end{figure}

\newpage

\begin{figure}[h!]
\centering
\includegraphics[width=0.7\textwidth]{Photo/CleanShot 2025-12-13 at 19.41.42@2x.png}
\caption{Случай некорректной записи}
\label{fig:contact-added6}
\end{figure}

\begin{figure}[h!]
\centering
\includegraphics[width=0.7\textwidth]{Photo/CleanShot 2026-01-11 at 13.33.25.png}
\caption{Случай некорректной записи}
\label{fig:contact-added5}
\end{figure}

\subsubsection{Use‑case 2: Редактирование существующего контакта}

\textbf{Цель:} пользователь изменяет данные выбранного контакта.

\textbf{Основной сценарий (корректный):}
\begin{enumerate}
    \item Пользователь выбирает строку в таблице.
    \item Нажимает кнопку «Редактировать».
    \item В диалоге изменяет необходимые поля.
    \item Нажимает «ОК».
    \item Изменения отображаются в таблице.
\end{enumerate}

\begin{figure}[h!]
\centering
\includegraphics[width=0.6\textwidth]{Photo/CleanShot 2025-12-13 at 19.50.24@2x.png}
\caption{Диалог редактирования выбранного контакта}
\label{fig:edit-dialog}
\end{figure}

\textbf{Некорректные сценарии:}
\begin{itemize}
    \item Пользователь нажимает «Редактировать» без выбранной строки (см. рисунок \ref{fig:contact-edit1}) -> предупреждение.
    \item Вводит некорректный email (см. рисунок \ref{fig:contact-added3}) -> предупреждение.
\end{itemize}

\begin{figure}[h!]
\centering
\includegraphics[width=0.6\textwidth]{Photo/CleanShot 2026-01-11 at 13.35.42.png}
\caption{Некорректный сценарий редактирования}
\label{fig:contact-edit1}
\end{figure}

\newpage

\subsubsection{Use‑case 3: Удаление контакта}

\textbf{Цель:} пользователь удаляет выбранный контакт.

\textbf{Основной сценарий:}
\begin{enumerate}
    \item Пользователь выбирает строку.
    \item Нажимает кнопку «Удалить».
    \item Подтверждает удаление в диалоговом окне -> контакт исчезает из таблицы.
    \item Пользователь нажимает «Нет» в диалоге подтверждения -> контакт остаётся.
\end{enumerate}

\begin{figure}[h!]
\centering
\includegraphics[width=0.7\textwidth]{Photo/CleanShot 2025-12-13 at 19.57.50@2x.png}
\caption{Диалог подтверждения удаления контакта}
\label{fig:delete-confirm}
\end{figure}

\textbf{Некорректные сценарии:}
\begin{itemize}
    \item Нажатие «Удалить» без выбранной строки (см. рисунок \ref{fig:delete-confirm}) -> предупреждение.
\end{itemize}

\newpage

\begin{figure}[h!]
\centering
\includegraphics[width=0.7\textwidth]{Photo/CleanShot 2026-01-11 at 13.44.33.png}
\caption{Диалог подтверждения удаления контакта}
\label{fig:delete-confirm}
\end{figure}

\subsubsection{Use‑case 4: Поиск по выбранному столбцу}

\textbf{Цель:} пользователь фильтрует контакты по введённому тексту.

\textbf{Корректный сценарий:}
\begin{enumerate}
    \item Пользователь вводит текст в поле поиска.
    \item Выбирает столбец (например, «Фамилия»), как показано на рисунке \ref{fig:search-example}.
    \item Таблица динамически скрывает строки, не содержащие совпадений.
\end{enumerate}

\begin{figure}[h!]
\centering
\includegraphics[width=0.7\textwidth]{Photo/CleanShot 2025-12-13 at 20.03.41@2x.png}
\caption{Результат поиска по выбранному столбцу}
\label{fig:search-example}
\end{figure}

\newpage

\textbf{Некорректные сценарии:}
\begin{itemize}
    \item Поиск по пустому тексту (см. рисунок \ref{fig:search-example2}) -> должны отображаться все строки.
    \item Поиск по столбцу «Телефоны» с несколькими номерами (см. рисунки \ref{fig:search-example3} и \ref{fig:search-example4}) -> проверяется каждый номер.
\end{itemize}

\begin{figure}[h!]
\centering
\includegraphics[width=0.7\textwidth]{Photo/CleanShot 2026-01-11 at 13.50.14.png}
\caption{Некорректный сценарий поиска}
\label{fig:search-example2}
\end{figure}

\begin{figure}[h!]
\centering
\includegraphics[width=0.7\textwidth]{Photo/CleanShot 2025-12-13 at 20.08.08@2x.png}
\caption{Некорректный сценарий поиска}
\label{fig:search-example3}
\end{figure}

\begin{figure}[h!]
\centering
\includegraphics[width=0.8\textwidth]{Photo/CleanShot 2025-12-13 at 20.10.48@2x.png}
\caption{Некорректный сценарий поиска}
\label{fig:search-example4}
\end{figure}

\subsubsection{Use‑case 5: Сохранение данных в файл}

\textbf{Корректный сценарий:}
\begin{enumerate}
    \item Пользователь нажимает «Сохранить».
    \item Файл успешно создаётся или перезаписывается.
    \item Появляется сообщение об успешном сохранении (см. рисунок \ref{fig:save-success}).
\end{enumerate}

\begin{figure}[h!]
\centering
\includegraphics[width=0.7\textwidth]{Photo/CleanShot 2025-12-13 at 20.20.50@2x.png}
\caption{Сообщение об успешном сохранении данных}
\label{fig:save-success}
\end{figure}

\newpage

\subsubsection{Use‑case 6: Загрузка данных из файла}

\textbf{Корректный сценарий:}
\begin{enumerate}
    \item Пользователь нажимает «Загрузить».
    \item Файл открывается.
    \item Контакты корректно отображаются в таблице (см. рисунок \ref{fig:load-success2}).
\end{enumerate}

\begin{figure}[h!]
\centering
\includegraphics[width=0.8\textwidth]{Photo/CleanShot 2025-12-13 at 20.22.56@2x.png}
\caption{Записи в файле}
\label{fig:load-success1}
\end{figure}

\begin{figure}[h!]
\centering
\includegraphics[width=0.8\textwidth]{Photo/CleanShot 2026-01-11 at 14.02.44.png}
\caption{Таблица после загрузки данных из файла}
\label{fig:load-success2}
\end{figure}

\subsubsection{Сортировка по выбранным столбцам}

Приложение поддерживает сортировку данных по любому столбцу таблицы. 
Сортировка выполняется автоматически при нажатии пользователем на заголовок столбца. 
Повторное нажатие изменяет направление сортировки (по возрастанию или по убыванию). 
При этом корректно сохраняется связь между строкой таблицы и реальным индексом контакта, 
что обеспечивает правильную работу функций редактирования и удаления.\\

Ниже приведены скриншоты, демонстрирующие работу сортировки.

\begin{figure}[h!]
\centering
\includegraphics[width=0.6\textwidth]{Photo/CleanShot 2025-12-13 at 20.14.32@2x.png}
\caption{Сортировка по фамилии}
\label{fig:search-example4}
\end{figure}

\begin{figure}[h!]
\centering
\includegraphics[width=0.6\textwidth]{Photo/CleanShot 2025-12-13 at 20.16.51@2x.png}
\caption{Сортировка по дате рождения}
\label{fig:search-example4}
\end{figure}

\begin{figure}[h!]
\centering
\includegraphics[width=0.6\textwidth]{Photo/CleanShot 2025-12-13 at 20.18.31@2x.png}
\caption{Сортировка по электронной почте}
\label{fig:search-example4}
\end{figure}

\newpage
