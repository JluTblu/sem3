\section*{Заключение}
\addcontentsline{toc}{section}{Заключение}

В ходе курсовой работы разработан калькулятор большой конечной арифметики $\langle Z_8^8; +, * \rangle$ на основе малой конечной арифметики $\langle Z_8; +, * \rangle$. Калькулятор поддерживает четыре действия: сложение, вычитание, умножение и деление с остатком.\\

\textbf{Реализованная функциональность}\\

Архитектура программы состоит из четырех модулей: constants для хранения констант кольца, utils для таблиц действий, arithmetic для базовых действий над символами, BigFiniteNumber для действий над многоразрядными числами.

Система таблиц действий построена на основе правила «+1» для варианта 48. Таблица сложения содержит 512 записей для всех комбинаций операндов и переносов.

Интерфейс позволяет выполнять действия через меню и запускать тесты проверки кольца.

Тестирование проверяет коммутативность сложения и умножения, ассоциативность сложения и умножения, дистрибутивность умножения относительно сложения, свойство умножения на нулевой элемент.

Деление реализовано алгоритмом деления столбиком с вычислением частного и остатка. Остаток всегда неотрицательный.\\

\textbf{Особенности реализации}\\

Малая арифметика реализует действия над односимвольными элементами. Все действия построены на базовом действии «+1».

Большая арифметика работает с многоразрядными числами. Сложение и вычитание выполняются поразрядно с управлением переносами и заемами.

Отрицательные числа обрабатываются через анализ знаков операндов. Действия сводятся к работе с абсолютными значениями с последующей установкой знака результата.

Деление отрицательного числа на положительное выполняет коррекцию частного и остатка для соблюдения условия неотрицательности остатка.\\

\textbf{Недостатки программы}\\

Умножение и деление имеют сложность $O(N \cdot M)$ из-за реализации через многократное сложение. Для чисел размера 8 разрядов это приводит к выполнению до 64 действий сложения на одно умножение.

Максимальная длина чисел фиксирована в коде константой MAX\_DIGITS. Изменение требует перекомпиляции программы.

Таблица сложения занимает 512 записей в памяти. При увеличении размера кольца память растет кубически.\\

\textbf{Масштабируемость}\\

Программа может быть расширена добавлением действий возведения в степень, нахождения НОД, вычисления НОК через НОД.

Возможно увеличение максимальной длины чисел через параметризацию константы MAX\_DIGITS и динамическое выделение памяти для результатов действий.

Интерфейс может быть дополнен графическим режимом работы (GUI) для удобства использования.
