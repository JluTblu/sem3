\section*{Введение}
\addcontentsline{toc}{section}{Введение}

Курсовая работа посвящена реализации калькулятора большой конечной арифметики на основе малой конечной арифметики. Работа выполнена на языке программирования C++17.

Задача состоит в построении калькулятора для конечного коммутативного кольца с единицей $\langle Z_8; +, * \rangle$, где операции определены над словами длины до 8 символов. Базовые действия реализованы через малую арифметику $\langle Z_8; +, * \rangle$, в которой задано правило +1.

В рамках работы реализованы:

Система правил кольца на основе заданного правила +1.

Модуль малой арифметики для действий над односимвольными элементами.

Модуль большой арифметики для многоразрядных чисел с управлением переносами и контролем переполнения.

Интерактивный интерфейс калькулятора с меню действий.

Набор тестов для проверки математических свойств.

Калькулятор поддерживает четыре действия: сложение, вычитание, умножение и деление с остатком. Все вычисления выполняются в рамках заданного конечного кольца с ограничением на максимальное количество разрядов.